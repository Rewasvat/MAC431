\input texbase

\titulo{Exercício Programa 2 - Fase II}
\materia{MAC0431 - Introdução à Computação Paralela e Distribuida}

\aluno{Fernando Omar Aluani}{6797226}
\aluno{Jefferson Serafim Ascaneo}{6431284}

\begin{document}
\cabecalho

\section{Descrição do Problema}

Hoje em dia muitos desenvolvedores usam sistemas de controle de versão para gerenciar
seus projetos, e entre esses sistemas, um que é amplamente utilizado é o \textit{Git}.
Uma das funcionalidades do Git é a possibilidade de exibir um relatório de informações 
de um repositório, listando os commits feitos, quem é o autor de cada commit, as alterações
realizadas, e mais.

Como pode existir grande variabilidade dos métodos e locais de trabalho dos desenvolvedores,
notar quais pessoas estão trabalhando mais ou menos torna-se um desafio. No entanto, o Git
pode nos ajudar fornecendo informações detalhadas sobre o repositório, possibilitando a
análise do trabalho de cada desenvolvedor.

Além disso, em certas metodologias de desenvolvimento de software é importante que os 
commits tenham um número reduzido de modificações, permitindo a integração contínua
do software sendo desenvolvido e um menor número de conflitos. Um outro ponto interessante
é a análise da estabilidade do código, e um indicador para isso é a quantidade de linhas
de um arquivo modificadas por commit. Um arquivo que sofre, em média, uma grande quantidade
de modificações é mais propenso a erros de programação por conter código mais recente e que
ainda não foi executado em um grande número de cenários. Todas estas informações podem ser
extraídas dos relatórios fornecidos pelo Git.

\subsection{Formato da Entrada de Dados}

O relatório do Git é um arquivo texto contendo a descrição de cada commit feito no repositório
de forma ordenada, do commit mais recente para o mais antigo. Tal relatório é dado pelo comando
\textbf{git log}, que contém diversos parâmetros para ajustar o formato do relatório.

Para o nosso projeto, é esperado que a descrição de cada commit no relatório tenha o seguinte
formato:

\begin{tiny}
\begin{verbatim}
<commit hash code> <author email>
<número de linhas adicionadas>       <número de linhas removidas>       <caminho/nome do arquivo modificado>
<número de linhas adicionadas>       <número de linhas removidas>       <caminho/nome do arquivo modificado>
<número de linhas adicionadas>       <número de linhas removidas>       <caminho/nome do arquivo modificado>
...
\end{verbatim}
\end{tiny}

Para tal, o seguinte comando é usado: 
\begin{verbatim}
  git log --numstat --pretty=format:"%H %ae"
\end{verbatim}

Dessa entrada, extraímos as seguintes estatísticas:
\begin{itemize}
\item Número de commits por desenvolvedor;
\item Média, desvio padrão e total de linhas modificadas de cada desenvolvedor;
\item Média, desvio padrão e total de linhas modificadas por commit;
\item Média, desvio padrão e total de linhas modificadas de cada arquivo.
\end{itemize}

%escrever um relatório descrevendo detalhes da implementação em Java (especialmente do Map/Reduce) e pontos positivos/negativos encontrados durante a implementação.
\section{Implementação}

\subsection{Map}
O Map basicamente coleta as amostras de dados do arquivo de entrada. Ele recebe um par $(chave, valor)$
de \textit{Text} (classe do Hadoop representando um texto, como uma \textit{string}), e coleta pares
$(chave, valor)$ onde a chave é uma string com o nome do dado, e valor é um \textit{double} com o seu valor.

A chave do par recebido é o hash code do commit, enquanto o valor é o próprio texto do commit, 
separado do arquivo de entrada. O Map então processa esse texto do commit, lendo linha por linha do
texto e analisando cada linha para extrair os dados que vamos usar, que são:
\begin{itemize}
  \item Número de linhas modificadas (adicionadas + removidas) para cada arquivo;
  \item Número de commits feitos pelo autor;
  \item Número de linhas modificadas no commit;
  \item Número de linhas modificadas pelo autor;
\end{itemize}

\subsection{Reduce}
O Reduce pega os pares de dados coletados pelo Map e calcula as estatísticas (soma, média e desvio
padrão) para cada conjunto de pares com a mesma chave. As estatísticas são coletadas e compõem
a saída do programa, adicionando o tipo da estatística como prefixo da chave a qual ela se refere. 

O algoritmo para calcular as estatísticas é um simples algoritmo para cálculo de soma, média e
desvio padrão de uma lista de valores.

Há alguns casos onde o resultado do desvio padrão é \textit{NaN} (not-a-number), devido a como ele
é calculado. Nesses casos, o valor do desvio padrão é setado como 0.

\subsection{InputFormat}
O padrão do Hadoop para leitura do arquivo de entrada e distribuição dos dados dele para os vários
nós de processamento é ler linha-por-linha, e distribuir as linhas. Isto não funcionaria no nosso
programa, pois precisamos analisar a entrada de forma \textit{commit-por-commit}, e cada commit
é composto por várias linhas de texto do arquivo de entrada.

Portanto, derivamos a classe padrão do Hadoop que gerencia a leitura e distribuição dos dados do
arquivo de entrada, \textit{FileInputFormat}, e criamos a nossa \textit{CommitInputFormat} que 
separa o arquivo de entrada nos blocos relativos à cada commit, distribuindo tais blocos para
o Map.

\subsection{RecordReader}
O \textit{InputFormat} usa internamente uma classe do Hadoop chamada \textit{RecordReader} para
a leitura do arquivo de entrada e divisão em blocos. Para criar nosso \textit{CommitInputFormat}
também tivemos que implementar um \textit{CommitRecordReader}.

Ele lê linha-por-linha do arquivo de entrada, separando os blocos quando encontra a linha que
indica um novo commit.


\section{Conclusão}
Ao implementar a solução em Hadoop, nos deparamos com os seguintes pontos:
\subsection{Pontos Positivos}
\begin{itemize}
  \item Fácil implementar seu algoritmo;
\end{itemize}

\subsection{Pontos Negativos}
\begin{itemize}
  \item Complicado para entender a princípio;
  \item Difícil testar/\textit{debuggar} seu programa;
  \item Saída do programa é problemática, gerando uma pasta com alguns arquivos estranhos e dando
        erros se a pasta já existe;
\end{itemize}

\end{document}
