\input texbase

\titulo{Exercício Programa 2 - Fase I}
\materia{MAC0431 - Introdução à Computação Paralela e Distribuida}

\aluno{Fernando Omar Aluani}{6797226}
\aluno{Jefferson Serafim Ascaneo}{6431284}

\begin{document}
\cabecalho

\section{Descrição do Problema}

Hoje em dia muitos desenvolvedores usam sistemas de controle de versão para gerenciar
seus projetos, e entre esses sistemas, um que é amplamente utilizado é o \textit{Git}.
Uma das funcionalidades do Git é a possibilidade de exibir um relatório de informações 
de um repositório, listando os commits feitos, quem é o autor de cada commit, as alterações
realizadas, e mais.

Como pode existir grande variabilidade dos métodos e locais de trabalho dos desenvolvedores,
notar quais pessoas estão trabalhando mais ou menos torna-se um desafio. No entanto, o Git
pode nos ajudar fornecendo informações detalhadas sobre o repositório, possibilitando a
análise do trabalho de cada desenvolvedor.

Além disso, em certas metodologias de desenvolvimento de software é importante que os 
commits tenham um número reduzido de modificações, permitindo a integração contínua
do software sendo desenvolvido e um menor número de conflitos. Um outro ponto interessante
é a análise da estabilidade do código, e um indicador para isso é a quantidade de linhas
de um arquivo modificadas por commit. Um arquivo que sofre, em média, uma grande quantidade
de modificações é mais propenso a erros de programação por conter código mais recente e que
ainda não foi executado em um grande número de cenários. Todas estas informações podem ser
extraídas dos relatórios fornecidos pelo Git.

\subsection{Formato da Entrada de Dados}

O relatório do Git é um arquivo texto contendo a descrição de cada commit feito no repositório
de forma ordenada, do commit mais recente para o mais antigo.

A descrição de cada commit em si segue o formato:

\begin{tiny}
\begin{verbatim}
commit <hash code>
Author: <nome> <email>
Date:   <data e hora>

    <texto descrevendo o commit>

<número de linhas adicionadas>       <número de linhas removidas>       <caminho/nome do arquivo modificado>
<número de linhas adicionadas>       <número de linhas removidas>       <caminho/nome do arquivo modificado>
<número de linhas adicionadas>       <número de linhas removidas>       <caminho/nome do arquivo modificado>
...
\end{verbatim}
\end{tiny}

\section{Abordagem para Solução}

Processar um relatório do Git recebido como entrada, extraindo dele diversas estatísticas, como:
\begin{itemize}
\item Número de commits por desenvolvedor;
\item Média, desvio padrão e total de linhas modificadas por desenvolvedor;
\item Média e desvio padrão de linhas modificadas por commit;
\item Média e desvio padrão de linhas modificadas por arquivo.
\end{itemize}

A função Map irá receber uma chave identificando o commit, que é um hash único gerado pelo Git,
e o valor é todo o texto descrevendo o commit. A saída será uma série de chaves e valores,
contendo, por exemplo, o par \texttt{<"lindev\_" + desenvolvedor, número de linhas
modificadas neste commit>}, \\
ou o par \texttt{<"numcommits\_" + desenvolvedor, 1>}.

\begin{verbatim}
map(chave, texto):
    desenvolvedor = parseNomeDesenvolvedor(texto)
    // Commits por desenvolvedor
    collect("numcommits_" + desenvolvedor, 1)
    total_linhas = 0
    listaArquivos = parseArquivos(texto)
    for arquivo in listaArquivos:
        total_linhas += arquivo.linhas_modificadas
        // Linhas modificadas por arquivo
        collect("arq_" + arquivo.nome, arquivo.linhas_modificadas)
    // Linhas modificadas por desenvolvedor
    collect("lindev_" + desenvolvedor, total_linhas)
    // Linhas modificadas por commit
    collect("linhas_mod_por_commit", total_linhas)
\end{verbatim}

A função Reduce irá receber os pares produzidos pela função Map e, a partir deles, irá
calcular a média, desvio padrão e total de cada tipo de chave.

\begin{verbatim}
reduce(chave, valores):
    soma=0
    somaQuadrado = 0
    c = 0
    for v in valores:
        soma += v
        somaQuadrado += v*v
        c++

    media = soma/c
    collect(chave + "_media", media)
    collect(chave + "_desvpadr",
            sqrt(abs(somaQuadrado - media * soma) / c))
    collect(chave + "_total", soma)
\end{verbatim}
\end{document}
