\input texbase

\titulo{Exercício Programa 4}
\materia{MAC0431 - Introdução à Computação Paralela e Distribuida}

\aluno{Fernando Omar Aluani}{6797226}
\aluno{Jefferson Serafim Ascaneo}{6431284}

\begin{document}
\cabecalho

\section{Detalhes da Implementação}

Nós decidimos que nosso programa de \textit{matrix shift} executa de uma vez só
um \textit{row\underline{ }shift(R)} e um \textit{column\underline{ }shift(C)}.
Dessa forma, o código é simplificado e o programa em si pode ser usado como uma
ferramenta, executando-o várias vezes em cadeia para atingir uma sequência de 
\textit{shifts} desejada.

\subsection{O Algoritmo}
O algoritmo em si é bem simples: ele lê uma matriz passada ao programa, executa
de uma vez só uma operação de \textit{row\underline{ }shift(R)} e um 
\textit{column\underline{ }shift(C)} juntamente (já que a ordem entre elas não importa),
e salva a matriz resultante num outro arquivo, com o mesmo nome do arquivo original mais
o sufixo \textquotedblleft\textit{\underline{ }result}\textquotedblright.

Para tal nós seguimos o exemplo do \textbf{VectorAdd}, fazendo algumas modificações: 
criamos uma classe para representar uma matriz $M\times N$ e nos ajudar com as operações que executamos
com as matrizes (do lado da CPU); mudamos os parâmetros passados/recebidos do kernel da
GPU; e mais importante, mudamos a quantidade de \textit{work items} nos quais seriam executados
o kernel para que ele fosse executado $M\times N$ vezes, numa grade, de tal forma que o kernel fosse
executado para cada elemento da matriz.

Quanto ao kernel em si, também alteramos ele para receber os parâmetros que especificamos e 
aplicar um \textit{row\underline{ }shift(R)} e um \textit{column\underline{ }shift(C)} de uma vez
só. Juntando com o fato que o kernel é executado para cada elemento da matriz, o código nele fica bem simples:
\begin{verbatim}
unsigned int i = get_global_id(0);
unsigned int j = get_global_id(1);

unsigned int a = (i - (R)) % M; //row shift
unsigned int b = (j + (C)) % N; //column shift

matriz_resultante[a*N + b] = matriz_original[i*N + j];

\end{verbatim}

E desse jeito também os \textit{shifts} funcionam como mostrados no enunciado:
\begin{itemize}
  \item \textit{row\underline{ }shift(R)} tem um sentido \textquotedblleft para cima\textquotedblright,
        de tal forma que com $R>0$ uma linha irá \textquotedblleft subir\textquotedblright  na matriz;
  \item \textit{column\underline{ }shift(C)} tem um sentido \textquotedblleft para direita\textquotedblright,
        de tal forma que com $C>0$ uma coluna irá se deslocar para a direita na matriz;
\end{itemize}

Todo o código do programa (a pasta com os arquivos, como mostrado no enunciado) está incluso neste arquivo,
na pasta \textbf{MatrixShift}.

\subsection{Entrada}
O programa recebe três argumentos pela linha de comando ao ser executado, no formato:
\begin{center}
\textit{./MatrixShift <input-matrix> <row-shift-amount> <column-shift-amount>}
\end{center}
Onde:
\begin{itemize}
  \item \textit{input-matrix}: é o caminho para um arquivo de texto contendo a matriz de entrada. Cada linha do
        arquivo é uma linha da matriz, e em cada linha os números (inteiros ou reais) devem estar separados
        por espaços (espaços separam colunas);
  \item \textit{row-shift-amount}: número de \textit{shifts} de linhas a fazer. É o $R$ mostrado anteriormente 
        (\textit{row\underline{ }shift(R)});
  \item \textit{column-shift-amount}: número de \textit{shifts} de colunas a fazer. É o $C$ mostrado anteriormente 
        (\textit{column\underline{ }shift(C)});
\end{itemize}

Note que o programa supõe que o arquivo da matriz de entrada é bem formado. Isto é, que ele segue essa especificação
e não tem valores não-numéricos, nem linhas de tamanhos (quantidade de valores) diferentes, etc. Passar um arquivo
de entrada mal-formado irá fazer o programa \textit{crashar} com algum erro ou retornar resultados estranhos.

\subsection{Saída}
O programa tem duas saídas. Durante a execução ele imprime alguns valores na saída padrão, mostrando o andamento da
execução do mesmo, e a matriz de entrada e a resultante depois das operações.

Fora isso, ele também salva a matriz resultante em um arquivo com o mesmo nome do arquivo original de entrada mais
o sufixo \textquotedblleft\textit{\underline{ }result}\textquotedblright.

\section{Conclusão}
\subsection{Pontos Positivos}
\begin{itemize}
  \item Os exemplos, principalmente o \textbf{VectorAdd} ajudaram bastante a entender como executar o código na GPU.
        Só precisamos pesquisar coisas sobre OpenCL na internet para entender direito a assinatura de algumas funções
        do OpenCL;
  \item A \textbf{AMD APP SDK} também ajuda bastante a setar o ambiente de desenvolvimento. Sua instalação é simples
        e não tivemos problemas em compilar e rodar o programa até numa VM com Linux rodando sobre um computador com
        CPU Intel e GPU NVidia;
  \item Nossa decisão de como aplicar os \textit{shifts} deixou o algoritmo bem simples;
  \item Mensagens de erro quando o programa tenta compilar o kernel mas não consegue ajudaram a achar os problemas;
\end{itemize}

\subsection{Pontos Negativos}
Complicação em compilar o programa dentro da hierarquia de pastas da AMD APP SDK, e rodar ele em outra pasta
ainda nessa hierarquia. Felizmente com um pequeno script bash para copiar os arquivos e rodar os comandos
necessários de qualquer outra pasta no sistema resolveu esse problema.

Tipos dos parâmetros a serem passados para o kernel. Foi o maior problema que tivemos ao fazer o programa. Achavámos
que podiamos passar os argumentos de um jeito, mas ou não era possível, como passar \textit{int} sendo que tinha que
ser \textit{int*}; ou deixaria o código mais complexo, como passar a matriz no formato \textit{float**}. Depois
passamos a matriz como um grande vetor (\textit{float*}) e tratamos ele como uma matriz no código do kernel por simplicidade.



\end{document}
